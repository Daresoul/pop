\documentclass{article}
\usepackage{graphicx}
\usepackage[utf8]{inputenc}
\usepackage{listings}
\usepackage{color}
\usepackage{upquote}

\definecolor{bluekeywords}{rgb}{0.13,0.13,1}
\definecolor{greencomments}{rgb}{0,0.5,0}
\definecolor{redstrings}{rgb}{0.9,0,0}
 
\lstdefinelanguage{FSharp}%
{morekeywords={let, override, new, match, with, rec, open, module, namespace, type, of, member, % 
and, for, while, true, false, in, do, begin, end, fun, function, return, yield, try, %
mutable, if, then, else, cloud, async, static, use, abstract, interface, inherit, finally },
  otherkeywords={ let!, return!, do!, yield!, use!, var, from, select, where, order, by },
  keywordstyle=\color{bluekeywords},
  sensitive=true,
  basicstyle=\ttfamily,
	breaklines=true,
  xleftmargin=\parindent,
  aboveskip=\bigskipamount,
	tabsize=4,
  morecomment=[l][\color{greencomments}]{///},
  morecomment=[l][\color{greencomments}]{//},
  morecomment=[s][\color{greencomments}]{{(*}{*)}},
  morestring=[b]",
  showstringspaces=false,
  literate={`}{\`}1,
  stringstyle=\color{redstrings},
}

\title{10g rapport}
\author{Henrik Flindt\\Nicolas Dyhrman\\Adrian Joensen}
\date{\today}

\begin{document}
    \maketitle
    
    \section*{Wolf in Moose's Clothing}
    \textit{"A single type of bird can be called a goose, but more is geese, but the plural of moose is not meese, and finally one wolf becomes several wolves. English is like coding. Nobody really know why we do it that way we do, but people on the internet will yell at you for getting it wrong."} \newline -Based on common saying.
    \section{Manual}
   		fsharpi animalsSmall.fsi animalsSmall.fs animalsSmall.fsx
    
    \section{Design}
    	We use Jon's template.
    	\\
    	To make the order of animals random, we create a tuple list containing a symbol and an index.
    	\\
    \subsection{Two-lists makes a board}
    \subsection{Animals}
    \subsubsection{Moose}
    \subsubsection{Wolf}    

    \section{Implementation}
		\lstset{language=FSharp}

		\begin{lstlisting}
/// An animal is a base class. It has a position and a reproduction counter.
type animal (symb : symbol, repLen : int) =
  let mutable _reproduction = rnd.Next(1,repLen)
  let mutable _pos : position option = None
  let _symbol : symbol = symb

  member this.symbol = _symbol
  member this.position
    with get () = _pos
    and set aPos = _pos <- aPos
  member this.reproduction = _reproduction
  member this.updateReproduction () =
    _reproduction <- _reproduction - 1
  member this.resetReproduction () =
    _reproduction <- repLen

  override this.ToString () =
    string this.symbol
    "delete this"
		\end{lstlisting}
		
         
    \section{White Box Testing}
   	   \begin{tabular}{|c|c|c|}
   	   		\hline
   	   		Input & Expected & Result\\
   	   		\hline
   	   		&&\\
   	   		\hline
   	   \end{tabular}
    \section{Conclusion}
     

\end{document}
